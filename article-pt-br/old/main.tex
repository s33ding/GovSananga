\documentclass{article}
\usepackage[utf8]{inputenc}
\usepackage[brazil]{babel}

\title{Projeto: Gov Sananga - Mapeamento e Análise de Empresas na Cidade Estrutural com Machine Learning e OSINT}
\author{}
\date{}

\begin{document}

\maketitle

\section*{Objetivo}

Criar uma ferramenta chamada \textbf{Gov Sananga} para mapear e analisar empresas formais e informais na Cidade Estrutural, utilizando técnicas de Machine Learning, scraping e OSINT.

\section*{Nome e Simbolismo}

\begin{itemize}
    \item \textbf{Gov}: Refere-se ao apoio governamental na coleta e análise de dados.
    \item \textbf{Sananga}: Um colírio indígena que simboliza clareza e visão, representando a capacidade da ferramenta de revelar informações cruciais sobre a economia informal.
\end{itemize}

\section*{Arquitetura e Tecnologias}

\begin{itemize}
    \item \textbf{Kubernetes (K8s)}: A ferramenta usará K8s para orquestrar containers, garantindo escalabilidade e resiliência da aplicação. K8s gerenciará a implantação automatizada e eficiente de modelos e serviços.
    \item \textbf{GPT para Leitura de Imagens}: GPT será utilizado para analisar e interpretar as imagens capturadas pelo Google Street View, identificando e categorizando visualmente estabelecimentos comerciais.
    \item \textbf{Google API para Scraping}: Utilizaremos a Google Custom Search API e Google Maps API para coletar dados estruturados sobre empresas, complementando a análise das imagens. Essa abordagem permitirá a coleta de informações detalhadas diretamente de fontes confiáveis.
    \item \textbf{Terraform}: Terraform será empregado para definir e provisionar a infraestrutura necessária em nuvem, facilitando a automação da criação de ambientes, como clusters K8s e serviços de armazenamento.
\end{itemize}

\section*{Estudo de Caso na Cidade Estrutural}

\begin{itemize}
    \item \textbf{Coleta de Dados}: Extração de informações de empresas através do Google Maps e Google Street View, combinada com o uso de GPT para análise de imagens e Google API para scraping de dados estruturados.
    \item \textbf{Análise Comparativa}: Comparação dos dados coletados com registros oficiais, para avaliar o impacto econômico e social das empresas informais.
\end{itemize}

\section*{Resultados Esperados}

\begin{itemize}
    \item \textbf{Mapeamento Detalhado}: Criação de uma base de dados abrangente, incluindo empresas formais e informais.
    \item \textbf{Ferramenta Escalável}: Infraestrutura robusta e escalável utilizando K8s, Terraform e Google API, aplicável a outras regiões.
    \item \textbf{Insights Econômicos}: Geração de dados e análises que auxiliem na formulação de políticas públicas.
\end{itemize}

\section*{Conclusão}

O Gov Sananga busca trazer clareza à economia informal da Cidade Estrutural, utilizando tecnologias avançadas como K8s, GPT, Google API e Terraform para criar uma solução escalável, automatizada e eficiente.

\end{document}

